%%
%% Author: novitoll
%% 3/8/18
%%

% Preamble
\documentclass[11pt]{article}

% Packages
\usepackage{amsmath}
\usepackage{enumitem}

\title{CVT: Lecture 5}
\date{2018-03-08}
\author{Novitoll}

\begin{document}
    \maketitle
    \pagenumbering{arabic}

    \section{Taylor formula} \label{sec:taylorformula}

    \begin{align}
        f(x) = f(0) + O(1)
    \end{align}

    Given parameter, compute its derivative and so you can "convert" the original function to Taylor
    \[
        f'(x) = f'(0) + O(1) --> \int_{0}^{x} ... dx
    \]

    So basic Taylor formula is:
    \begin{multline*}
        f(x) = f(0) + f'(0)x + f''(0)\frac{x^2}{2} \\
            + f'''(0)\frac{x^3}{6} + f^{IV}(0)\frac{x^4}{24} + f^n(0)\frac{x^n}{n!} + O(x^n)
    \end{multline*}

    Example:
    \begin{gather*}
        f(x) = x^2 + 3x + 6 \\
        f(0) = 6; f'(0) = 3; f''(0) = 2; f'''(0) = 0; f^n(0) = 0, n > 2 \\
        Taylor: f(x) = 6 + 3x + 2\frac{x^2}{2}
    \end{gather*}

    \section{Optical flow in general} \label{sec:optflowgeneral}

    Given x, y coordinate of some pixels in image (assuming that we are dealing with the only RGB pixels) in given t time,
    we can process the next image's pixels x, y coordinate in some $t + 1$ time

    \begin{enumerate}
        \item Constant color
    \end{enumerate}

    \begin{gather*}
        I_1(x, y, t) \\
        I_2(x + u, y + v, t + 1) \\
        O = I(x + u, y + v, t + 1) - I(x, y, t)
    \end{gather*}

    Let's put it to Taylor formula, assuming that we will neglect with high order derivatives due to non-significant bias of pixels.

    \begin{enumerate}[resume]
        \item Small motion
    \end{enumerate}

    \begin{gather*}
        I(x + u, y + v) = I(x, y) + \frac{\partial{I}}{\partial{x}}u + \frac{\partial{I}}{\partial{y}}v
    \end{gather*}
    \begin{multline}
        O \approx I(x, y, t + 1) + \frac{\partial{I}}{\partial{x}}u + \frac{\partial{I}}{\partial{y}}v - I(x, y, t) \\
            \approx [I(x, y, t + 1) - I(x, y, t)] + .. \\
            \approx \frac{\partial{I}}{t} + \frac{\partial{I}}{\partial{x}}u + \frac{\partial{I}}{\partial{y}}v
    \end{multline}

    , which can be also exposed as $\frac{\partial{I}}{t} + \det{I} \times <u, v>$,
    e\.g\. time derivative + multiplication of image gradient and vector of motion u, v per x, y axis.

    \section{Lucas-Kanade optical flow (1981)} \label{sec:lucaskanade}

    Adding here additional assumption:

    \begin{enumerate}[resume]
        \item Uniform motion of pixels
    \end{enumerate}

    Assume that you have a pixel x and you take it as 3x3 matrix with surrounded other pixels,
    assuming that they are similar (in real life this is true for the object, probably not true for micro objects like
    bacteria)

    , then your formula (2) is true for 3x3 matrix, e.g.:

    \[
        \begin{bmatrix}
            I_x(p_1) & \dots & I_y(p_1) \\
            I_x(p_2) & \dots & I_y(p_2) \\
            \vdots & \ddots & \vdots \\
            I_x(p_9) & \dots & I_y(p_9)
        \end{bmatrix}
        \times
        \begin{bmatrix}
            u & v
        \end{bmatrix}
        = -
        \begin{bmatrix}
            I_t(p_1) \\
            I_t(p_2) \\
            \vdots   \\
            I_t(p_9)
        \end{bmatrix}
    \]

    Let's name the matrix in A, d, b:

    \[
        A_{9x2} \times d_{2x1} = B_{9x1}
    \]

    and now we want to

    \[
        minimize => \lVert{Ad - b}\rVert^2
    \]

    \subsection{Eigenvalue and eigenvector} \label{subsec:eigenvv}

    Recap eigenvalue and eigenvector of linear transformation:
    \begin{gather*}
        T(\vec{v}) = \lambda\vec{v} \\
        , where \ \vec{v} - eigenvector, \ \lambda - eigenvalue, e.g. some scalar
    \end{gather*}



\end{document}